\title{LZW}
\author{Vicente González Ruiz}
\maketitle
\tableofcontents

\section{Basics}

\begin{itemize}
\item
  In 1984, Terry A. Welch proposed the
  \href{https://en.wikipedia.org/wiki/Lempel\%E2\%80\%93Ziv\%E2\%80\%93Welch}{LZW
    algorithm}~\cite{welch1984technique}, an improved version of the
  LZ78 algorithm that does not writes raw symbols (\(k\) fields) to
  the code-stream.
\item
  LZW was selected as the encoding engine for the
  \href{https://en.wikipedia.org/wiki/GIF}{GIF (Graphics Interchange
  Format)}, and for the compressor
  \href{https://en.wikipedia.org/wiki/Compress}{\texttt{compress}}.
\item
  Initially, the dictionary is filled with the \(2^k\) possible symbols
  (\emph{roots}), that are stored in the first entries (for 1-byte
  symbols: \(0\cdots255\)).
\end{itemize}

\section{Encoder}

\begin{enumerate}
\tightlist
\item
  \(w\leftarrow\) first input symbol.
\item
  While the input is not exhausted:
  \begin{enumerate}
  \tightlist
  \item
    \(k\leftarrow\) next input symbol.
  \item
    If \(wk\) is found in the dictionary, then:
    \begin{enumerate}
    \tightlist
    \item
      \(w\leftarrow\) address of \(wk\) in the dictionary.
    \end{enumerate}
  \item
    Else:
    \begin{enumerate}
    \tightlist
    \item
      Output \(\leftarrow w\).
    \item
      Insert \(wk\) in the dictionary.
    \item
      \(w\leftarrow k\).
    \end{enumerate}
  \end{enumerate}
\end{enumerate}

\section*{Example}
\myfig{graphics/LZW_encoding_example}{4cm}{400}

\section{Decoder}

\begin{enumerate}
\tightlist
\item
  \(c\leftarrow\) first input code-word.
\item
  Output \(c\).
\item
  \(c'\leftarrow c\).
\item
  While the input is not exhausted:
  \begin{enumerate}
  \tightlist
  \item
    \(c\leftarrow\) next input code-word.
  \item
    \(w\leftarrow c'\).
  \item
    If \(c\) is found in the dictionary, then:
    \begin{enumerate}
    \tightlist
    \item
      Output string\((c)\).
    \end{enumerate}
  \item
    Else:
    \begin{enumerate}
    \tightlist
    \item
      Output string\((w)\).
    \item
      Output \(k\).
    \end{enumerate}
  \item
    \(k\leftarrow\) first symbol of the last output.
  \item
    Insert \(wk\) in the dictionary.
  \item
    \(c'\leftarrow c\).
  \end{enumerate}
\end{enumerate}

\section*{Example}
\myfig{graphics/LZW_decoding_example}{6cm}{600}

\section{Lab: LZW}
\href{https://nbviewer.jupyter.org/github/vicente-gonzalez-ruiz/LZW/blob/master/LZW.ipynb}{IPython notebook}

\bibliography{text-compression}
